%% For double-blind review submission
\documentclass[sigplan,10pt,review]{acmart}\settopmatter{printfolios=true}
%% For single-blind review submission
%\documentclass[acmlarge,review]{acmart}\settopmatter{printfolios=true}
%% For final camera-ready submission
%\documentclass[acmlarge]{acmart}\settopmatter{}

%% Note: Authors migrating a paper from PACMPL format to traditional
%% SIGPLAN proceedings format should change 'acmlarge' to
%% 'sigplan,10pt'.


%% Some recommended packages.
\usepackage{booktabs}   %% For formal tables:
                        %% http://ctan.org/pkg/booktabs
\usepackage{subcaption} %% For complex figures with subfigures/subcaptions
                        %% http://ctan.org/pkg/subcaption
\usepackage{amsmath}

% Mine!
\usepackage{alltt}
\usepackage{bcprules}
\usepackage{listings}
\usepackage{color}

\lstdefinelanguage{ivo}{
  morekeywords={%
    case,catch,class,config,data,default,do,finally,for,fun,import,export,let,module,native,returns,throw,trait,try,type,var,where,yields},
  %otherkeywords={..,<-,->,:=},
  sensitive=true,
  morecomment=[l]{//},
  %morecomment=[l]{#},
  morecomment=[n]{/*}{*/},
  morestring=[b]",
  morestring=[b]',
  morestring=[b]"""
}[keywords,comments,strings]

\definecolor{mygreen}{rgb}{0,0.6,0}
\definecolor{mygray}{rgb}{0.5,0.5,0.5}
\definecolor{mymauve}{rgb}{0.58,0,0.82}

\lstset{frame=tb,
  language=ivo,
  aboveskip=3mm,
  belowskip=3mm,
  showstringspaces=false,
  keepspaces=true,
  columns=flexible,
  keywordstyle=\color{blue},
  commentstyle=\color{mygreen},
  stringstyle=\color{mymauve},
  frame=none,
  breaklines=true,
  breakatwhitespace=true,
  tabsize=2,
  numbers=left,                    % where to put the line-numbers; possible values are (none, left, right)
  numbersep=5pt,                   % how far the line-numbers are from the code
  numberstyle=\tiny\color{mygray}, % the style that is used for the line-numbers
  basicstyle=\ttfamily }

\renewcommand\c[1]{\relax\ifmmode\mathtt{#1}\else\lstinline[language=ivo,mathescape=true]{#1}\fi}
\newcommand\bnf{\,\,|\,\,}
\newcommand\ty{\,:\,}
\newcommand\fn{\rightarrow}
\newcommand\SB[1]{[\![#1]\!]}
\newcommand\eat[1]{}

\makeatletter\if@ACM@journal\makeatother
%% Journal information (used by PACMPL format)
%% Supplied to authors by publisher for camera-ready submission
\acmJournal{PACMPL}
\acmVolume{1}
\acmNumber{1}
\acmArticle{1}
\acmYear{2017}
\acmMonth{1}
\acmDOI{10.1145/nnnnnnn.nnnnnnn}
\startPage{1}
\else\makeatother
%% Conference information (used by SIGPLAN proceedings format)
%% Supplied to authors by publisher for camera-ready submission
\acmConference[Scala 2017]{Eighth ACM SIGPLAN Symposium on Scala}{October 22--23, 2017}{Vancouver, BC, Canada}
\acmYear{2017}
\acmISBN{978-x-xxxx-xxxx-x/YY/MM}
\acmDOI{10.1145/nnnnnnn.nnnnnnn}
\startPage{1}
\fi


%% Copyright information
%% Supplied to authors (based on authors' rights management selection;
%% see authors.acm.org) by publisher for camera-ready submission
\setcopyright{none}             %% For review submission
%\setcopyright{acmcopyright}
%\setcopyright{acmlicensed}
%\setcopyright{rightsretained}
%\copyrightyear{2017}           %% If different from \acmYear


%% Bibliography style
\bibliographystyle{ACM-Reference-Format}
%% Citation style
%% Note: author/year citations are required for papers published as an
%% issue of PACMPL.
\citestyle{acmauthoryear}   %% For author/year citations



\begin{document}

%% Title information
\title{A Scala framework for supercompilation}         %% [Short Title] is optional;
                                        %% when present, will be used in
                                        %% header instead of Full Title.
%\titlenote{with title note}             %% \titlenote is optional;
                                        %% can be repeated if necessary;
                                        %% contents suppressed with 'anonymous'
%\subtitle{Subtitle}                     %% \subtitle is optional
%\subtitlenote{with subtitle note}       %% \subtitlenote is optional;
                                        %% can be repeated if necessary;
                                        %% contents suppressed with 'anonymous'


%% Author information
%% Contents and number of authors suppressed with 'anonymous'.
%% Each author should be introduced by \author, followed by
%% \authornote (optional), \orcid (optional), \affiliation, and
%% \email.
%% An author may have multiple affiliations and/or emails; repeat the
%% appropriate command.
%% Many elements are not rendered, but should be provided for metadata
%% extraction tools.

%% Author with single affiliation.
\author{Nathaniel Nystrom}
% \authornote{with author1 note}          %% \authornote is optional;
                                        %% can be repeated if necessary
% \orcid{nnnn-nnnn-nnnn-nnnn}             %% \orcid is optional
\affiliation{
  % \position{Position1}
  \department{Faculty of Informatics}              %% \department is recommended
  \institution{Universit{\`a} della Svizzera italiana}
  \streetaddress{Via Giuseppe Buffi 13}
  \city{Lugano}
  %\state{State1}
  \postcode{6900}
  \country{Switzerland}
}
\email{nate.nystrom@usi.ch}          %% \email is recommended

%% Paper note
%% The \thanks command may be used to create a "paper note" ---
%% similar to a title note or an author note, but not explicitly
%% associated with a particular element.  It will appear immediately
%% above the permission/copyright statement.
%\thanks{with paper note}                %% \thanks is optional
                                        %% can be repeated if necesary
                                        %% contents suppressed with 'anonymous'


%% Abstract
%% Note: \begin{abstract}...\end{abstract} environment must come
%% before \maketitle command
\begin{abstract}
  
Partial evaluation is the well-known idea of specializing a program
on some of its inputs. 
Supercompilation is a related technique 
that has been proposed for program optimization, especially
for function languages.
Supercompilation extends traditional partial evaluation by recognizing repeated computations and generalizing to ensure the partial evaluator terminates.
We describe a framework for building partial evaluators and supercompilers 
directly from an interpreter.
The user specifies the interpreter using rewrite rules
and the framework 
handles termination checking, generalization, and residualization.
We demonstrate the approach by implementing a supercompiler for
JavaScript.

\end{abstract}


%% 2012 ACM Computing Classification System (CSS) concepts
%% Generate at 'http://dl.acm.org/ccs/ccs.cfm'.
\if 0
\begin{CCSXML}
<ccs2012>
<concept>
<concept_id>10011007.10011006.10011008</concept_id>
<concept_desc>Software and its engineering~General programming languages</concept_desc>
<concept_significance>500</concept_significance>
</concept>
<concept>
<concept_id>10003456.10003457.10003521.10003525</concept_id>
<concept_desc>Social and professional topics~History of programming languages</concept_desc>
<concept_significance>300</concept_significance>
</concept>
</ccs2012>
\end{CCSXML}

\ccsdesc[500]{Software and its engineering~General programming languages}
\ccsdesc[300]{Social and professional topics~History of programming languages}
%% End of generated code
\fi


%% Keywords
%% comma separated list
\keywords{supercompilation, partial evaluation, language frameworks}  %% \keywords is optional


%% \maketitle
%% Note: \maketitle command must come after title commands, author
%% commands, abstract environment, Computing Classification System
%% environment and commands, and keywords command.
\maketitle


\newcommand\R[1]{\langle\!\langle#1\rangle\!\rangle}

\section{Introduction}
\label{sec:intro}

Partial evaluation is the well-known 
idea of specializing a program on some of its inputs~\cite{jones}.
Supercompilation is a partial evaluation technique that
attempts to evaluate programs as much as possible
at compile time~\cite{turchin}.
The name \emph{supercompilation} is a portmanteau of supervisor and compiler.
The supercompiler works by monitoring the execution of a (partial) evaluator.
When evaluation cannot proceed from a given program state, the
state is \emph{split} and evaluation is performed recursively. The
supercompiler conservatively identifies states from which continued evaluation
and splitting will not terminate and \emph{generalizes}
these states, ensuring the supercompiler terminates and 
producing a residual program
that (hopefully) implements the original program more efficiently.

Supercompilation has received considerable attention in the functional
programming 
literature because it subsumes common optimizations like deforestation~\cite{
  supercompilation-by-eval,supercompilation-blah}.
However, the design and implementation of a practical supercompiler is
elusive.
It is
not yet in mainstream use in any compiler because of the following
practical considerations: (1) supercompilation has a large compile time
compared to traditional optimization, and (2) supercompilation can generate
exponentially large code.
Attempts to address these issues have been made in
other work \cite{taming-code-explosion}, but not yet applied to 
mainstream languages such as Haskell or Scala.
To address these issues, more experiments need be done with different
supercompilation algorithms.

To further this research,
we describe a Scala framework for implementing 
partial evaluators and supercompilers.
To build a supercompiler for a given language, the programmer
need only provide the following:
\begin{enumerate}
\item Abstract syntax trees for the terms of the language.
\item An interpreter implemented as small-step rewrite rules
between equivalent program states. 
The interpreter can be implemented by term rewriting or as
an abstract machine~\cite{danvy,cesk}, allowing, for instance,
mutable variables or unstructured control-flow to be more easily supported.
\item Extensions of the interpreter to 
support \emph{residual states}. These are states that represent 
the residual program being constructed.
Stepping a residual state produces another residual state, and
ultimately results in the residual program.
\item A \emph{split} function to decompose a ``stuck'' program state into
one or more other states.
This function is used, for instance, when branching on an abstract value,
to return two states one which assumes the value was true, the other false.
After these two states are evaluated, the stuck state can be reassembled
into a rebuilt state from which more evaluation can be done.
\item A \emph{generalization} function 
that takes two states, producing a new state that generalizes the two.
\end{enumerate}

The framework provides the supercompilation algorithm and data structures
as well as code for detecting possible nontermination,
performing variable substitutions, and generalizations. Framework
code can be used as is, but supercompiler instances can override
the framework implementation for more precision. 

\paragraph*{Outline}
The rest of this paper is organized as follows.
Section~\ref{sec:nutshell} sketches how supercompilation works using
a small example.
In the Section~\ref{sec:framework}, we describe the interface
and implementation of our supercompiler framework.
Section~\ref{sec:cesk} describes the process of extending an abstract machine interpreter
to construct a supercompiler.
We use our framework to implement a supercompiler for JavaScript,
described in Section~\ref{sec:js}.
% Section~\ref{sec:experiments} describes some experiments.
Related work is discussed in Section~\ref{sec:related}.
Finally, we conclude in Section~\ref{sec:conclusions} with a discussion of
future work.

\section{Supercompilation in a nutshell}
\label{sec:nutshell}

To get an idea of what supercompilation does and how it works, consider the
\c{map} function on lists.\footnote{Examples are presented in Scala syntax.
The map example is due to \citeN{supercompilation-by-eval}.}
\begin{verbatim}
  def map[A,B](f: A => B, as: List[A]): List[A] =
    as match {
      case Nil => Nil
      case b::bs => f(b)::map(f, bs)
    }
\end{verbatim}

Consider the call \c{map(_ + 1, zs)}.
Supercompiling this call constructs a specialized version of \c{map}
that increments each element of the list by 1.

Evaluation is called \emph{driving} in the supercompiler literature.
We illustrate driving this expression by showing the state of the evaluator
as a pair $(e; k)$. The expression $e$ is the expression being
evaluated and $k$ is the continuation of the expression.
The initial state of the evaluation is the pair of the expression and the
empty continuation:
  \[
    \c{map(_ + 1, zs)};~k = \bullet
  \]
Since the list argument is statically unknown, 
we pass an \emph{abstract value} into the 
\c{map} function.
To keep the presentation simple, we substitute the actual arguemnts for the
formal parameters.
  \[
    \begin{array}{l}
      \c{zs~match~\{} \\ 
    \c{~~~case~Nil \Rightarrow Nil}  \\
      \c{~~~case~b::bs \Rightarrow  (b+1)::map(\_+1, bs)~\}};~k = \bullet
  \end{array}
  \]
The body of \c{map}
performs a pattern match on the abstract value \c{zs}).
However, since \c{zs} is not statically known, the evaluator has reached an impasse.
Here the supercompiler steps in and 
\emph{splits} the evaluation state into two: oone where \c{zs} is \c{Nil}, the
other where it is not.

Assuming \c{zs} is \c{Nil}, we arrive immediately in this state:
    \[
      \c{Nil};~k = \bullet
    \]
Here driving stops.
The supercompiler produces a residual expression $\R{\c{Nil}}$.
we write
$\R{e}$ for a residual expression $e$ to be output 
by the supercompiler. For purposes of
this example, residual expressions may be considered values.
Applying a continuation to a residual expression produces a larger residual
expression.\footnote{In
practice, residual expressions as values
must be treated a \emph{linear} values to avoid work duplication.
For instance, we cannot simply substitute a residual expression
for a function parameter since that would duplicate the expression
for each occurrence of the parameter in the function body.}

On the other hand, if \c{zs} is not \c{Nil}, we drive as follows:
    \begin{align*}
       \c{(b+1)::map(\_+1, bs)};~& k = \bullet \\
       \downarrow&\\
       \c{(b+1)};~& k = \c{\bullet::map(\_+1, bs)} \\
       \downarrow&\\
       \c{b};~& k = \c{(\bullet+1)::map(\_+1, bs)} 
    \end{align*}
Here again driving is stuck because \c{b} is unknown.
Unlike before, we cannot split the evaluator state, instead we \emph{rebuild},
generating a residual
expression $\R{\c{b}}$:
       \[
       \R{\c{b}};~ k = \c{(\bullet+1)::map(\_+1, bs)}
     \]
After continued driving:
    \begin{align*}
       \R{\c{b+1}};~& k = \c{\bullet::map(\_+1, bs)} \\
       \downarrow&\\
       \c{map(\_+1, bs)};~& k = \c{\R{b+1}::\bullet} 
    \end{align*}
the supercompiler arrives in a state where
the focus expression is a call to \c{map}.
Indeed, the arguments are a renaming of the original function arguments.
At this point, the supercompiler \emph{folds}.
It generates a function stub that generalizes the two calls:
\begin{verbatim}
  def map1(as: List[A]): List[A] = ...
\end{verbatim}
and replaces the focus expression with a residual call to that function.
To construct the body of the new function, it continues driving the residual
expression.
    \begin{align*}
      \c{\R{map1(bs)}};~& k = \c{\R{b+1}::\bullet} \\
       \downarrow&\\
      \c{\R{(b+1)::map1(bs)}};~& k = \bullet 
    \end{align*}
Here again, driving has stopped.
After splitting on \c{zs}, the evaluator has produced two residual expressions
\c{Nil} and $\c{\R{(b+1)::map1(bs)}}$.
The supercompiler reassembles the pattern match from the split states and produces 
the final residual program:
\begin{verbatim}
  def map1(as: List[A]): List[A] =
    as match {
      case Nil => Nil
      case b::bs => (b+1)::map1(bs)
    }

  map1(zs)
\end{verbatim}

\section{Our framework}
\label{sec:framework}

Our focus is on building supercompilers for program specialization and
optimization. Supercompilation has been used in other applications, in
particular for theorem proving and function inversion~\cite{mrsc}.
These applications have different tradeoffs. For instance, in theorem proving
applications, the program being supercompiled is 
typically a pure functional program with call-by-name semantics,
and efficiency and size of the generated code is often not a big concern.


\subsection{The interpreter}

To implement a supercompiler for a given language, the user must provide an
interpreter for the language. The framework requires the interpreter be
implemented by stepping from one program state to another. This implementation
supports different modes of interpretation, in particular term rewriting
and abstract machines.

An interpreter implements the \c{Interpreter} trait shown in Figure~\ref{fig:interpreter}.
The interface is functional: the state of the interpreter should 
be maintained only in a \c{State} value, on which the trait is parameterized.
The trait requires a \c{step} method
that advances the interpreter to the next state.
Importantly, 
this method need not be total:
the interpreter can get ``stuck'' in a state because of missing
information. If the interpreter becomes stuck, the supercompiler can
still make further progress, as outlined below.
Indeed, we take advantage of this property to implement
a supercompiler for JavaScript. Indeed, our JavaScript interpreter does not
support all features of JavaScript, but using our framework,
any JavaScript program can be supercompiled.
A more complete interpreter implementation leads to
a more effective supercompiler, however, completeness is not required.

When the \c{step} method returns \c{None}, the supercompiler calls \c{split}
on the current interpreter state.
If successful, \c{split} returns a list of states the supercompiler should 
attempt to drive. It also returns an \c{Unsplitter}, which is a function
that attempts to recombine the results of the split states.

If both \c{step} and \c{split} fail, the supercompiler calls \c{rollback}.
This takes the history of states the supercompiler executed (up to the
previous split), and returns a new history, which replaces the old.
This Orwellian rewriting of history is needed by some interpreter
implementations to help construct residuals program in later states.

The \c{fold} method attempt to fold the current state (the last
state in the history) with a previous state. A new generalized state
is returned. Fold is called before each attempt to \c{step}.

Finally the \c{whistle} function checks for possible nontermination
of a history.
If it returns \c{true} (``the whistle blows''), the supercompiler stops driving as if \c{step}
returned \c{None}. Our framework provides several whistle implementations
that can be used by the interpreter. Different whistle implementations are
a topic of ongoing research.

\begin{figure*}
  \begin{verbatim}
trait Interpreter[State] {
  // Step to the next state, if possible.
  def step(s: State): Option[State]

  // Split the current state, if possible.
  def split(s: State): Option[(List[State], Unsplitter[State])]

  type History = List[State]

  // Replace the history with a new history. Returns Nil on failure.
  def rollback(h: History): History

  // Fold the current state (the last state in the history)
  // with a state in the history. If successful, returns a
  // new state that generalizes the two folded states.
  def fold(h: History): Option[State]

  // Check if we should stop driving.
  def whistle(h: History): Boolean
}
  \end{verbatim}
  \caption{Interprter trait}
  \label{fig:interpreter}
\end{figure*}

\subsection{The supercompilation algorithm}

We present supercompilation itself as evaluation on an abstract machine.
The states of the machine are shown in Figure~\ref{fig:sc-data}.
To distinguish between supercompiler states and interpreter state we refer
to the latter as metastates.

Each metastate contains the current interpreter state, a (possibly truncated or
revised) history of previous states and a continuation. 
There are four kinds of metastate: \c{Drive}, \c{Split}, \c{Rebuild},
and \c{Done}.
Continuations are stacks of \c{Meta}\-\c{Frame}. The continuation manages the
splitter of the supercompiler.
The \c{RunSplit} frame
contains the list of unprocessed split states, completed split histories,
as well as the root state that caused the split, its history, and an \c{Unsplitter}
function that takes the completed split histories and attempts to rebuild
the root state.

\begin{figure*}
  \begin{verbatim}
type History = List[State]
type MetaCont = List[MetaFrame]

sealed trait MetaState
case class Drive(state: State, history: History, k: MetaCont) extends MetaState
case class Split(state: State, history: History, k: MetaCont) extends MetaState
case class Rebuild(state: State, history: History, k: MetaCont) extends MetaState
case class Done(state: State) extends MetaState

sealed trait MetaFrame
case class RunSplit(pending: List[State], complete: List[History],
                    root: State, rootHistory: History,
                    unsplit: Unsplitter[State]) extends MetaFrame
  \end{verbatim}
  \caption{Supercompiler data types}
  \label{fig:sc-data}
\end{figure*}

The supercompiler implementation is shown in Figure~\ref{fig:sc}.
The \c{sc} function takes the current metastate and a history of
metastates, computes the next metastate and then calls itself recursively
if the metastate is not \c{Done}.

If the metastate is \c{Drive}, the supercompiler tries to fold
the current history. If successful it enters the \c{Rebuild}
metastate with the new folded history.
Otherwise, if the whistle blows, indicating 
the current interpreter state might lead to nontermination,
the supercompiler enters the \c{Split} metastate.
Otherwise, the supercompiler tries to take a step, splitting on failure.

In the \c{Split} metastate, the supercompiler attempts to split
the current interpreter state.
If successful, it \c{Drives} the first split state, pushing a continuation to
evaluate the other split states.
On failure, the supercompiler \c{Rebuilds}.

In the \c{Rebuild} metastate, the supercompiler
either starts driving the next pending split state 
or, if the split is complete, attempts to reassemble the root
state of the split. Three outcomes are possible: either a
new root state is produced, or the unsplit function forces
another split, or the unsplit fails.
In the last case, the supercompiler attempts to rollback
the current history. If the rollback succeeds, driving continues,
otherwise the split (meta)frame is popped and 
the root of the split is rebuilt.

If in the \c{Rebuild} state and the continuation is empty,
a rollback is attempted. If the rollback fails, the supercompiler
terminates.

\begin{figure*}
  \begin{verbatim}
def sc(ms: MetaState, mhistory: List[MetaState]): MetaState = {
  val next = ms match {
    // Try to drive
    case Drive(s, h, k) =>
      interp.fold(s::h) match {
        case Some(s1::h1) =>
          Rebuild(s1, h1, k)   // fold successful
        case _ if (interp.whistle(s::h) || metaWhistle(ms::mhistory)) =>
          Split(s, h, k)
        case _ =>
          interp.step(s) match {
            case Some(s1) => Drive(s1, s::h, k)
            case None     => Split(s, h, k)     } }

    // Try to split
    case Split(s, h, k) =>
      interp.split(s) match {
        case Some((first::rest, unsplit)) if ! metaWhistle(ms::mhistory) =>
          Drive(first, Nil, RunSplit(rest, Nil, s, h, unsplit)::k)
        case _ => Rebuild(s, h, k) }

    // Run the next split state
    case Rebuild(s, h, RunSplit(next::pending, complete, s0, h0, unsplit)::k) =>
      Drive(next, Nil, RunSplit(pending, complete :+ (s::h), s0, h0, unsplit)::k)

    // All the split states are done. Try to reassemble
    case Rebuild(s, h, RunSplit(Nil, complete, s0, h0, unsplit)::k) =>
      unsplit(complete :+ (s::h)) match {
        case UnsplitOk(s1) =>
          Drive(s1, h0, k)
        case Resplit(states, unsplit) if metaWhistle(ms::mhistory) =>
          Rebuild(s0, h0, k)
        case Resplit(first::rest, unsplit) =>
          Drive(first, Nil, RunSplit(rest, Nil, s0, h0, unsplit)::k)
        case _ =>
          interp.rollback(s::h) match {
            case s1::h1 => Drive(s1, h1, k)
            case Nil    => Rebuild(s0, h0, k)   }

    case Rebuild(s, h, Nil) =>
      interp.rollback(s::h) match {
        case s1::h1 => Drive(s1, h1, Nil)
        case Nil    => Done(s) }

    case Done(s) =>
      // Finished
      return Done(s)
  }

  sc(next, ms::mhistory) // Keep going
}
  \end{verbatim}
  \caption{Supercompiler algorithm}
  \label{fig:sc}
\end{figure*}


\subsection{The whistle}

In the supercompilation literature, the \emph{whistle}
is a subroutine that detects if the supercompiler
might not terminate.
This can happen, of course, when evaluating an actual infinite loop, but also 
when evaluating loops where the loop condition is statically unknown, or
when evaluating recursive calls. Indeed nontermination occurs more often
during partial evaluation than normal evaluation because of missing
information.

If the whistle blows, the supercompiler
stops driving and splits the current state.
Our supercompiler uses two whistles: the first operates
on interpreter states (implemented by \c{Interpreter.whistle})
and ensures termination of driving. The second operates on states of the supercompiler
itself and ensures termination of splitting.

Whistles are an ongoing topic of research~\cite{whistles64}.
Typically, the whistle is based on an \emph{well-quasi-ordering} (WQO) of
interpreter states.
Indeed, any WQO on states can be used as a whistle.
States of the supercompiler are stored in a history.
When a new state is entered, it is compared to previous states
in the history. If the new state is greater than or equal a previous state,
according to the WQO, then
there is a potential infinite sequence of states
the supercompiler will pass through. The supercompiler halts driving or
splitting to avoid running forever.

Various WQOs have been used in the supercompiler literature.
Typically, they are based on a homeomorphic embedding of
states~\cite{whistles64}.
Our framework provides several whistles: a whistle that
never blows (thus allowing nontermination), a whistle that 
blows when the state size becomes too large or when the history of states becomes too
long.
The default whistle provided by the framework uses a \emph{tag-bag}.
In our implementation, 
the tag-bag of a state is a bag (multiset) of 
Scala class names in the state.
One state is less than another if it has the same set of tags
in the bag, but the bag is smaller.
\begin{align*}
  s \preceq s'
   \iff&
  \mathit{set}(\mathit{tagbag}(s))
  =
  \mathit{set}(\mathit{tagbag}(s')) \\
  &
  \wedge
  |\mathit{tagbag}(s)|
  \le
  |\mathit{tagbag}(s')|
\end{align*}

The implementation works for arbitrary state implementations, using the Kiama
library~\cite{kiama} to traverse data structures, collecting class names.
Users of the framework can override the tagging operation for more precision,
for instance by including constant values occurring in the state as additional tags.

\subsection{Folding and generalization}

Another component of the supercompilation is the \c{fold} function.
Like the whistle,
folding compares the current state with previous states in the history.
If two states are identical up to renaming, the states can be \emph{generalized}.
For instance, in the example from Section~\ref{sec:nutshell},
the \c{map} function is called initally as
\c{map(\_ + 1, zs)}
and then later
\c{map(\_ + 1, bs)}.
When supercompilation is complete the residual of the first call (with \c{zs})
will be the body of a function that maps \c{\_ + 1} over the free
variable \c{zs}.
We can generalize these two calls by introducing a new function
that takes \c{zs} as a parameter. 
Folding replaces the call in the current state
with a call to the new function, and then builds the residual function body.
The history is rolled back to the previous state, with the original call
replaced with a call to the new function.

The framework provides methods for generalizing terms.
A generalization of two terms
replaces differing subterms of the two terms with a variable.
The \emph{most-specific generalization}
(MSG) 
performs the fewest substitutions possible~\cite{supercompilation-blah}.

\begin{verbatim}
trait MSG[Term, Var <: Term] {
  type Subst = Map[Var, Term]
  def freshVar: Var
  def msg(t1: Term, t2: Term):
    Option[(Term, Subst, Subst)] = ...
}
\end{verbatim}

To use this functionality, the user implements the \c{MSG} trait
providing a method to generate fresh variable terms.
The \c{msg} method returns not only the MSG term, but also two substitutions, one 
that when applied to the MSG returns the first term \c{t1}
and the other that when applied to the MSG returns \c{t2}.
The framework provides methods for building and applying substitutions.
The MSG \emph{always} exists, but the \c{msg} method fails if the MSG
is just a variable, meaning the head of the two terms
is of a different shape.
If the two terms are identical upto renaming, they are
required to be folded. This is one of the key distinctions between supercompilation and simple
partial evaluation.


\section{Supercompiling imperative languages}

Our focus in building the framework, and indeed our primary use case,
is for program optimization and specialization.
To that end our framework provides support for 
implementing interpreters (and therefore supercompiler
driving and rebuilding) for imperative languages.
The framework provides traits for implementing interpreters
in CESK-machine style. 
These traits need not be used in the interpreter implementtion,
however they help support
typical features of imperative languaes.
In particular, the framework provides code for managing an abstract store,
and utilities for 
splitting, unsplitting, and
residualization of loops, functions, and exceptions.

\subsection{CESK interpreters}

In a CESK interpreter, the interpreter state is given by a 4-tuple:
(control, environment, store, kontinuation [sic])~\cite{cesk}.
We implement the interpreter using
``apply, eval, continue'' style of Danvy~\cite{danvy}.
An eval state $\c{Ev}(e, \rho, \sigma, k)$
has a control (or focus) expression $e$, an environment $\rho$,
which maps names to memory locations, a store $\sigma$, which maps
memory locations to values, and a continuation $k$.
An \c{Ev} state
dispatches on the focus expression $e$ and
changes the focus of the interpreter to a subexpression of $e$, pushing a new continuation frame.
A continue state $\c{Co}(v, \sigma, k)$
has a value $v$ as its focus, dispatches on the continuation, performing an
operation. 
Our implementation treats apply states (which apply operations)
as a special case of continue states.

For example, the evaluate the assignment \c{x = 1+2}, the interpreter goes through the following
states:
\begin{align*}
& \c{Ev}(\c{x = 1+2}, \rho, \sigma, \bullet) \\ 
& \c{Ev}(\c{1+2}, \rho, \sigma, (\c{x = \bullet},\rho)::\bullet) \\
& \c{Ev}(\c{1}, \rho, \sigma, (\c{\bullet+2},\rho)::(\c{x = \bullet},\rho)::\bullet) \\
& \c{Co}(\c{1}, \sigma, (\c{\bullet+2},\rho)::(\c{x = \bullet},\rho)::\bullet) \\
& \c{Ev}(\c{2}, \rho, \sigma, (\c{1+\bullet},\rho)::(\c{x = \bullet},\rho)::\bullet) \\
& \c{Co}(\c{2}, \sigma, (\c{1+\bullet},\rho)::(\c{x = \bullet},\rho)::\bullet) \\
& \c{Co}(\c{3}, \sigma, (\c{x = \bullet},\rho)::\bullet) \\
& \c{Co}(\c{3}, \sigma[\c{x}\mapsto\c{3}], \bullet)
\end{align*}

To these two kinds of state, we add an
$\c{Unwinding}(j, \sigma, k)$ state, which is like a \c{Co}, but has a jump statement
$j$
as the focus. \c{Unwinding} states are used to implement unstructured control
flow like return, throw, break, and continue statements.
Evaluating an \c{Unwinding} state pops continuations until reaching
the jump target (e.g., the caller frame if the jump is a return).
\c{try}--\c{finally} statements can be supported by evaluating them during the
stack unwinding.

We also add a 
$\c{Residual}(e, \sigma, k)$ state, which contains 
a residual expression $\R{e}$. Stepping \c{Residual} state
generally produces another \c{Residual} state with a larger residual
expression.

All states in the framework implement the following trait:
\begin{verbatim}
trait State {
  def step: Option[State]
  def split: Option[(List[State], Unsplitter[State])]
}
\end{verbatim}


\subsection{The abstract store}

The main difficulty with supercompilation of imperative languages 
is simulating the store.
In our framework, 
the store maps memory locations to either values or 
to a special \emph{unknown} value.
If, during evaluation, a memory location maps to the unknown value,
driving stops, triggering a split.

When constructing a \c{Residual} state, the effect of the residual expression
on the store must be simulated. Any variables assigned in the
residual expression are ``forgotten'' by mapping their locations to unknown.
Simulating a call conservatively forgets all values in the store except variables
in the environment.

If a branching expression (for instance an \c{if}) is split,
the then- and else-branches of the \c{if} are evaluated with the same store.
When reassembling an \c{if}, the resulting stores must be \emph{merged}.
Two stores $\sigma_1$ and $\sigma_2$ are merged by preserving mappings where the same location
is mapped to the same value and mapping other locations to the unknown value.
Essentially this forgets any information about locations that differ in the
two stores.

To evaluate the branches more precisely, we extend the store
with information known about the branch condition.
For instance if branching on a variable \c{x}, we evaluate the then-branch with \c{x} set to \c{true} and the
else-branch with \c{x} set to \c{false}.
This optimization is often done in functional language
supercompilers~\cite{supercompilation-by-eval,mitchell-supero},
which propagate pattern matches into case arms.

Loops present another compilation.
To split and reassemble loops we must simulate the effect of a possibly
infinite sequence of loop interations on the store.
Consider, for instance, a while loop \c{while~(e)~s}.
When split, the supercompiler 
evaluates the sequence \c{e;~s}. During reassembly, if the final store after evaluating the sequence differs from
the initial store, the stores are merged and the loop is re-split.
The \c{e;~s} sequence is re-evaluated and again
reassembly is attempted. Re-splitting until the final state equals the (merged) initial store.
Repeated re-splitting is guaranteed to terminate because merging only forgets
information: eventually a fixed point is reached. Effectively, this approach discards from the heap any values
that are not loop invariant.\footnote{The store could be made more precise after
the loop by simulating a \c{false} loop condition, as is done
with the condition for else-branches of \c{if}. However, \c{break} statements complicate this optimization.}

To support reassembly of split states, methods are provided for inspecting the
histories of states. For instance, after splitting an $n$-argument call expression,
say \c{f(x+1,2+3,4+z)}, the three operands are evaluated to produce arguments for 
a rebuilt call to \c{f}.
Because evaluation must capture store effects, the splitter does not return
three different states, but rather just one state that evaluates the operands
in sequence. Evaluation starts with an empty continuation.
Evaluation of each operand grows, then shrinks the continuation. 
After each operand but the last finishes evaluation, the continuation of the resulting state is
the same. After the last operand is evaluated, the continuation is again empty.
The framework provides utility functions for extracting from the final states for each operand
from a history of states.

\eat{

consider splitting on the call \c{1+2+x)}.
We would like the residual to be \c{3+x}.
Evaluation of this expression goes through the following states:
\begin{align*}
        \c{1+2+x};~& k = \c{\bullet+x} \\
        \c{1+2};~& k = \c{\bullet+x} \\
        \c{1};~& k = \c{(\bullet+2)+x} \\ 
        \c{2};~& k = \c{(1+\bullet)+x} \\ 
        \c{3};~& k = \c{\bullet+x} \\ 
        \c{x};~& k = \c{3+\bullet} \\ 
        \c{\R{x}};~& k = \c{3+\bullet} \\ 
        \c{\R{3+x}};~& k = \c{\bullet} \\ 
\end{align*}

}

\eat{
\subsection{Residualizing the state}

One problem that occurs is how to residualize a term
containing an abstract memory location.
We take the following approach:
whenever we compute a location
(either by looking up a variable
in the environment or by looking up property in the store),
we annotate the location with the access path used to compute it.
The path is only used when the address is in the focus
of the machine state or when the address is stored in a continuation.

The path is only used to residualize the location.

For new objects, the path is a new object expression.
For local variables, it is the name of the variable.
For field accesses, it is the path computed for the object's location
plus the field name.

If stored in a continuation,
we need to ensure the path is not invalidated
before it is used.
When reifying the continuation, although the
path may have changed, between the time the path is put into the
continuation and the time the continuation is popped,
when the continuation (and therefore the path) is reified,
the path was valid.
}

\section{From a CESK machine to supercompilation}
\label{sec:cesk}

To get a better idea of how the framework is used, we defined
a simple imperative language and a CESK-like abstract machine
for the language.
We then extend the interpreter to handle residual states
and then describe how splitting and reassembly are implemented.
The syntax for our language is given in Figure~\ref{fig:imp-syntax}.
Most constructs are standard. The language has mutable variable,
records with mutable fields, and control-flow expressions.
Values include memory locations $\ell$.

\begin{figure}
\begin{align*}
    e &::= v \\
        & \bnf x \\
        & \bnf e_1 \oplus e_2 \\
        & \bnf \c{while}~(e_1)~e_2 \\
        & \bnf \c{if}~(e_0)~e_1~\c{else}~e_2 \\
        & \bnf \c{var}~x;~e \\
        & \bnf x = e \\
        & \bnf e_1;~e_2 \\
        & \bnf \{ \overline{x} = \overline{e} \} \\
        & \bnf e.x \\
        & \bnf e.x = e \\
    v &::= n \bnf \c{true} \bnf \c{false} \bnf \ell
\end{align*}
  \caption{Syntax for example imperative language}
  \label{fig:imp-syntax}
\end{figure}

\subsection{Driving}

The operational semantics of the language are given in Figure~\ref{fig:imp-cesk}.
As described above, machine states are either \c{Ev} or \c{Co} states.
Unlike with Felleisen and Friedman's CESK machine, continuations
are not stored in the store (because the language does not support first-class
continuations). 
The rules are largely standard.
The \c{var} epxression allocates a location and
adds \c{x} to the scope (rule \textsc{Ev-let}).
Assigning to a field 
either updates 
an existing field (\textsc{Co-setField})
or creates a new field (\textsc{Co-createField}).

\newcommand\then[2]{(\bullet;~#1, #2)}
\newcommand\branch[3]{(\c{if}~(\bullet)~#1~\c{else}~#2, #3)}
\newcommand\init[3]{(\bullet.#1 = #2, #3)}

\newcommand{\residual}[1]{\langle\!\langle{#1}\rangle\!\rangle}


\begin{figure*}
  \begin{center}
\begin{align*}
  \omit [\textsc{Ev-val}] &&
    \c{Ev}(v, \rho, \sigma, k)
    & \longrightarrow
    \c{Co}(v, \sigma, k)
\\
  \omit [\textsc{Ev-var}] &&
    \c{Ev}(x, \rho, \sigma, k)
    & \longrightarrow
    \c{Co}(\sigma(\rho(x)), \sigma, k)
\\
  \omit [\textsc{Ev-let}] &&
    \c{Ev}(\c{var}~x;~e, \rho, \sigma, k)
    & \longrightarrow
    \c{Ev}(e, \rho', \sigma', k) 
        \\ &&& (\rho' = \rho[x := \ell], \sigma' = \sigma[\ell := \c{false}], \ell~\mbox{fresh})
\\
  \omit [\textsc{Ev-op}] &&
    \c{Ev}(e_1 \oplus  e_2, \rho, \sigma, k)
    & \longrightarrow
    \c{Ev}(e_1, \rho, \sigma, (\bullet \oplus e_2,\rho)::k)
\\
  \omit [\textsc{Co-op1}] &&
    \c{Co}(v_1, \sigma, (\bullet \oplus e_2,\rho')::k)
    & \longrightarrow
    \c{Ev}(e_2, \rho', \sigma, (v_1\oplus \bullet,\rho')::k)
\\
  \omit [\textsc{Co-op2}] &&
    \c{Co}(v_2, \sigma, (v_1 \oplus \bullet,\rho')::k)
    & \longrightarrow
    \c{Co}(v_1 \oplus v_2, \sigma, k)
\\
  \omit [\textsc{Ev-seq}] &&
    \c{Ev}(e_1;~e_2, \rho, \sigma, k)
    & \longrightarrow
    \c{Ev}(e_1, \rho, \sigma, \then{e_2}{\rho}::k)
\\
  \omit [\textsc{Co-seq}] &&
    \c{Co}(v, \sigma, \then{e_2}{\rho'}::k)
    & \longrightarrow
    \c{Ev}(e_2, \rho', \sigma, k)
\\
  \omit [\textsc{Ev-if}] &&
    \c{Ev}(\c{if}~(e_0)~e_1~\c{else}~e_2, \rho, \sigma, k)
    & \longrightarrow
    \c{Ev}(e_0, \rho, \sigma, \branch{e_1}{e_2}{\rho}::k)
\\
  \omit [\textsc{Co-ifTrue}] &&
    \c{Co}(\c{true}, \sigma, \branch{e_1}{e_2}{\rho'}::k)
    & \longrightarrow
    \c{Ev}(e_1, \rho', \sigma, k)
\\
  \omit [\textsc{Co-ifFalse}] &&
    \c{Co}(\c{false}, \sigma, \branch{e_1}{e_2}{\rho'}::k)
    & \longrightarrow
    \c{Ev}(e_2, \rho', \sigma, k)
\\
  \omit [\textsc{Ev-while}] &&
    \c{Ev}(\c{while}~(e_1)~e_2, \rho, \sigma, k)
    & \longrightarrow
    \c{Ev}(\c{if}~(e_1)~(e_2;~ \c{while}~(e_1)~e_2)~\c{else}~\c{false}, \rho, \sigma, k)
\\
  \omit [\textsc{Ev-setVar}] &&
    \c{Ev}(x = e, \rho, \sigma, k)
    & \longrightarrow
    \c{Co}(\rho(x), \sigma, (\bullet = e, \rho)::k)
\\
  \omit [\textsc{Co-setVar}] &&
    \c{Co}(\ell, \sigma, (\bullet = e, \rho')::k)
    & \longrightarrow
    \c{Ev}(e, \rho', \sigma, (\ell = \bullet, \rho)::k)
\\
  \omit [\textsc{Co-assign}] &&
    \c{Co}(v, \sigma, (\ell = \bullet, \rho')::k)
    & \longrightarrow
    \c{Co}(v, \sigma[\ell := v], k)
\\
  \omit [\textsc{Ev-new}] &&
    \c{Ev}(\{ \overline{x} = \overline{e} \}, \rho, \sigma, k)
    & \longrightarrow
    \c{Co}(\ell, \sigma', (\init{\overline{x}}{\overline{e}}{\rho}::k)
        \\ &&& (\sigma' = \sigma[\ell := \c{\{\}}], \ell~\mbox{fresh})
\\
  \omit [\textsc{Co-init}] &&
    \c{Co}(\ell, \sigma, \init{x_0,\overline{x}}{e_0,\overline{e}}{\rho'}::k)
    & \longrightarrow
    \c{Ev}(\ell.x_0 = e_0, \rho', \sigma,
    \then{\ell}{\rho'}::\init{\overline{x}}{\overline{e}}{\rho'}::k)
\\
  \omit [\textsc{Co-obj}] &&
    \c{Co}(\ell, \sigma, \init{\cdot}{\cdot}{\rho'}::k)
    & \longrightarrow
    \c{Co}(\ell, \sigma, k)
\\
  \omit [\textsc{Ev-getField}] &&
    \c{Ev}(e.x, \rho, \sigma, k)
    & \longrightarrow
    \c{Ev}(e, \rho, \sigma, (\bullet.x, \rho)::k)
\\
  \omit [\textsc{Co-getField}] &&
    \c{Co}(\ell, \sigma, (\bullet.x_i, \rho')::k)
    & \longrightarrow
    \c{Ev}(\sigma(\ell_i), \rho', \sigma, k)
        \\ &&& (\sigma(\ell) = \{ \overline{x} = \overline{\ell}\})
\\
  \omit [\textsc{Ev-setField}] &&
    \c{Ev}(e_1.x = e_2, \rho, \sigma, k)
    & \longrightarrow
    \c{Ev}(e_1, \rho, \sigma, (\bullet.x = e_2, \rho)::k)
\\
  \omit [\textsc{Co-setField}] &&
    \c{Co}(\ell, \sigma, (\bullet.x_i = e, \rho')::k)
    & \longrightarrow
    \c{Ev}(e, \rho', \sigma, (\ell_i = \bullet, \rho)::k)
        \\ &&& (\sigma(\ell) = \{ \overline{x} = \overline{\ell}\})
\\
  \omit [\textsc{Co-createField}] &&
    \c{Co}(\ell, \sigma, (\bullet.x_0 = e, \rho')::k)
    & \longrightarrow
    \c{Ev}(e, \rho', \sigma', (\ell_0 = \bullet, \rho)::k)
        \\ &&& (
                \sigma(\ell) = \{ \overline{x} = \overline{\ell} \},
                \sigma' =
                \sigma[\ell := \{ x_0 = \ell_0, \overline{x} = \overline{\ell} \}
                       ][\ell_0 := \c{false}], \ell_0~\mbox{fresh})
\\
\end{align*}
  \end{center}

\caption{Evaluation semantics for the example language}
\label{fig:imp-cesk}
\end{figure*}

To implement the interpreter for this language 
a \c{step} method is implemented that matches the current state
and returns the next state (wrapped in an \c{Option}) if one of the above
rules applies.
If one the rules does not match, the supercompiler performs splitting.
Splitting is described below.

\subsection{Rollback and residual states}

If splitting fails, a \c{Residual} state is created with the stuck focus
as if the following two rules were added:
\begin{align*}
     \c{Co}(v, \sigma, k)
        & \longrightarrow
        \c{Residual}(\R{v}, \sigma', k) \\ 
     \c{Ev}(e, \rho, \sigma, k)
        & \longrightarrow
        \c{Residual}(\R{e}, \sigma', k) \\
\end{align*}
In these ``rule'', $\sigma'$ is the store obtained by simulating the focus
expression on $\sigma$.
In the implementation, this rewriting is done
by the \c{rollback} method of the \c{Interpreter} trait (see Figure~\ref{fig:interpreter}).

One problem that occurs is how to residualize a term
containing an abstract memory location since these are not legal surface
syntax and exist only as part of the interpreter state.
Given the state $\c{Co}(\ell, \sigma, k)$ we can walk back through
the history to the earlier state $\c{Ev}(e, \sigma', k)$ that created the location $\ell$
and residualize $\ell$ as $\c{Residual}(\R{e}, \sigma', k)$.
We assume a function $\mathit{reify}(\ell)$ that returns the access path used
to compute $\ell$.

Another detail is that we need to residualize variable declarations to ensure
they are declared in the residual. We modify \textsc{Ev-let} as follows, and
add a rule to pop the \c{var}-rebuilding continuation if not needed:
\[
  \begin{array}{l} \c{Ev}(\c{var}~x;~e, \rho, \sigma, k)
     \longrightarrow
    \c{Ev}(e, \rho, \sigma', (\c{var}~x;~\bullet, \rho)::k)
        \\ \qquad  (\rho' = \rho[x := \ell], \sigma' = \sigma[\ell := \c{false}], \ell~\mbox{fresh})
      \\
    \c{Co}(v, \sigma, (\c{var}~x;~\bullet, \rho)::k)
     \longrightarrow
    \c{Co}(v, \sigma, k) \\
  \end{array}
\]

Once we have residual states we want to compute with them. The step function
for \c{Residual} states is shown in Figure~\ref{fig:imp-cesk-res}.
In general, for each \c{Co} rule, there should be a \c{Re} rule that
grows the residual state.
Rules \textsc{Re-let-free} and \textsc{Re-let-bound} ensure 
that any free variables in the residual are declared.
Several rules update the store. The function $\mathit{sanitize}(\sigma, x)$
removes all fields named \c{x} from the store.
The $\sigma \sqcap \sigma'$ is the merge of two stores.

Stepping with residual rules always terminates: each rule consumes one 
frame of the continuation.
This ensures the supercompiler itself terminates.
When the whistle blows or driving or splitting cannot be done,
the interpreter transitions to a residual state, producing the residual
program in a finite number of steps.

The rules make no attempt to be clever, for instance, evaluating both branches
of the \c{if}. This is the job of the splitter.
If the splitter works well, residual rules are used for only short continuations,
leaving the generated residual small.

\begin{figure*}
  \begin{center}
\begin{align*}
  \omit [\textsc{Re-op}] &&
    \c{Re}(\residual{e_2}, \sigma, (v_1 \oplus \bullet,\rho')::k)
    & \longrightarrow
    \c{Re}(\residual{v_1 \oplus e_2}, \sigma, k)
\\
  \omit [\textsc{Re-seq}] &&
    \c{Re}(\residual{e_1}, \sigma, \then{e_2}{\rho'}::k)
    & \longrightarrow
    \c{Re}(\residual{e_1;~e_2}, \sigma, k)
\\
  \omit [\textsc{Re-if}] &&
    \c{Re}(\residual{e}, \sigma, \branch{e_1}{e_2}{\rho'}::k)
    & \longrightarrow
    \c{Re}(\residual{\c{if}~(e)~e_1~\c{else}~e_2}, \sigma \sqcap \sigma', k)
\\
  \omit [\textsc{Re-assign}] &&
    \c{Re}(\residual{e},\sigma, (\ell = \bullet, \rho')::k)
    & \longrightarrow
    \c{Re}(\residual{p = e}, \sigma[\ell := \bot], k)
        \\ &&& (p = \mathit{reify}(\ell))
\\
  \omit [\textsc{Re-getField}] &&
    \c{Re}(\residual{e},\sigma, (\bullet.x, \rho')::k)
    & \longrightarrow
    \c{Re}(\residual{e.x}, \sigma, k)
\\
  \omit [\textsc{Re-setField}] &&
    \c{Re}(\residual{e_1},\sigma, (\bullet.x = e_2, \rho')::k)
    & \longrightarrow
    \c{Re}(\residual{e_1.x = e_2}, \mathit{sanitize}(\sigma, x), k)
\\
  \omit [\textsc{Re-let-free}] &&
    \c{Re}(\residual{e},\sigma, (\c{var}~x;~\bullet, \rho')::k)
    & \longrightarrow
    \c{Re}(\residual{\c{var}~x;~e}, \sigma, k) 
        \\ &&& (x \in \mathit{FV}(e))
\\
  \omit [\textsc{Re-let-bound}] &&
    \c{Re}(\residual{e}, \sigma, (\c{var}~x;~\bullet, \rho')::k)
    & \longrightarrow
    \c{Re}(\residual{e}, \sigma, k) 
        \\ &&& (x \not\in \mathit{FV}(e))
\\
\end{align*}
  \end{center}

\caption{Evaluation semantics for example language with residuals}
\label{fig:imp-cesk-res}
\end{figure*}
    
\subsection{Splitting}

The splitter is responsible for further evaluating subexpressions from
a stuck state. The interpreter can get stuck either because no rule applies
or because the whistle signals that driving should stop.

This splitter is implemented by providing the \c{split} method of the \c{Interpreter}
trait. In the CESK interpreter, each \c{State} has a \c{split} method
which returns a list of \c{State} and an \c{Unsplitter} function, shown
in see Figure~\ref{fig:unsplit}.

\begin{figure*}
\begin{verbatim}
object Unsplit {
  type Unsplitter[State] = List[List[State]] => UnsplitResult[State]

  sealed trait UnsplitResult[State]
  case class Resplit[State](ss: List[State], unsplit: Unsplitter[State]) extends UnsplitResult[State] {
    require(ss.nonEmpty)
  }
  case class UnsplitOk[State](s: State) extends UnsplitResult[State]
  case class UnsplitFail[State]() extends UnsplitResult[State]
}
\end{verbatim}
\caption{Unsplitter interface}
\label{fig:unsplit}
\end{figure*}

Branches are the most interesting case for the splitter.
If the interpreter is stuck in the following state:
\[
    \c{Co}(v, \sigma, \branch{e_1}{e_2}{\rho'}::k)
\]
The splitter returns two states:
\[
    \c{Ev}(e_1, \rho', \sigma_1, \bullet)
    \quad
    \quad
    \mbox{and}
    \quad
    \quad
    \c{Ev}(e_2, \rho', \sigma_2, \bullet)
\]
Note that the continuation of these states is empty.
This is a design choice: including the continuation $k$ in the split states
might provide better performance, but would also lead to exponential code
growth. Essentially, this would put the continuation of the rest of the
program after the \c{if} in each branch of the \c{if}.

The two states are provided with different stores.
$\sigma_1$ is obtained by adding information about the branch test
$v$ to the store, assuming $v$ is \c{true}.
$\sigma_2$ is obtained by assuming $v$ is \c{false}.

After driving these two states,
the end result should be two \c{Residual} states.
The unsplitter attempts to produce a new \c{Ev} or \c{Co} state from these states.
For instance, if the two resulting states are identical values, the \c{if}
can be discarded and replaced with the value, the two result states are merged
and driving can continue.
In the worst case the unsplitter
reconstructs the \c{if}, producing a new \c{Residual} state.

If the language were extended with unstructured control flow, for instance a
\c{break} statement, the unsplitter has another option.
If may be that one branch gets stuck trying to unwind the continuation
to exit a loop, while the other branch completes normally. This can happen
with an expression like:
\begin{verbatim}
  while (..) {
    if (..)
      break
    else
      work()
    more()
  }
\end{verbatim}
The interpreter gets stuck if the continuation of the \c{break} runs out before
the enclosing loop is found.
In this case, rather than failing, we can \emph{extend} the continuation of
both branches, for instance, adding \c{more()} and the next loop iteration
to the continuation of the branch. The state is then \emph{resplit} with the
new continuations. This would result in the following residual program:
\begin{verbatim}
  while (..) {
    if (..)
      break
    else {
      work()
      more()
    }
  }
\end{verbatim}

\eat{

Soundness
if the program would evaluate to termination in full environment and store,
the program will evaluate to termination in a partial environment and store.

Progress
if a step can be taken with a full env,
a step can be taken in a partial env with a residual value.

the machine can always take a step.
If we arrive at a state from which no step can be taken, 
we residualize the focus 

In BB, each state maps to a name.
Each state corresponds to just an expression being evaluated,
not a complete program.
Each state maps to a name.
If a state is encountered again, we replace the state
with a call to the function of that name.

when we start supercompiling a state, record it with a name,
then replace it with the optimized binding




\section{Residualizing the state}

One problem that occurs is how to residualize a term
containing an abstract memory location.
We take the following approach:
whenever we compute a location
(either by looking up a variable
in the environment or by looking up property in the store),
we annotate the location with the access path used to compute it.
The path is only used when the address is in the focus
of the machine state or when the address is stored in a continuation.

The path is only used to residualize the location.

For new objects, the path is a new object expression.
For local variables, it is the name of the variable.
For field accesses, it is the path computed for the object's location
plus the field name.

If stored in a continuation, 
we need to ensure the path is not invalidated
before it is used.
When reifying the continuation, although the 
path may have changed, between the time the path is put into the
continuation and the time the continuation is popped,
when the continuation (and therefore the path) is reified,
the path was valid.
        XXX need to argume better


To compute a term from a state, we residualize the focus
and evaluate the continuation, always residualizing the focus
between each step.
The main problem is to residualize memory locations.
To do this, we compute for each location an access path for that 
location. The access path is computed by performing a breadth-first search through the store, starting with the locations in the environment.
While traversing the store, we build up an access path to the location, appending property accesses.
Using BFS gives us the shortest access path possible for the given
environment.
For this to work, we must ensure that each live location is reachable 
from the environment and not just from a continuation.
If a location is unreachable from the environment, it must either be garbage (in which case it should not be reified at all), or it must be reachable from
the continuation only.
In the latter case, the location must be a newly created object
that has yet to be stored in a variable.
We disallow reification of these locations: instead we step the
machine to a state where the new object is store in a variable
and then reify the state.

More simply: we do not reify states with locations unreachable from
from the environment. To ensure these do not exist, we introduce
local variables to ensure locations get stored immediately after creation.

We transform loc(0) into a local variable \c{loc_0}.
We replace assignments in the continuation 

We transform the program as follows.

\begin{verbatim}
        { x : e } --> t = {}; t.x = e; t

        // in this case the new object is reachable from either this
        // (in the body of the F constructor)
        // or from t at all points
        new F(e) --> t = new F(e); t

        // any object returned from f is made reachable again from t
        f() --> t = f(); t

        // we add t to the environment
\end{verbatim}

When are locations reified?

        when used in a binary expression and the other operand
        is unknown

        when used in a call and the function is unknown

        when used in a property lookup and the property is unknown

        we ensure that 

        reifying the branches of a conditional when the test is unknown

                x ? new C : new D

OR:
        every location stores its path
        just have to be careful of linearity
        when creating a location, store the residual for that location
        when updating a location, update the residual

                x ? new C : new D
        ->
                x ? { y = new C ; y }
                  : { z = new D ; z }


\section{Overall structure}

We traverse the JavaScript AST.

We add each (open) term to the environment as a named function.
We supercompile each function.
If we encounter a term again, we replace it with a call.
Or we inline the call.
Which terms do we add then?

The body of any lambda encountered is supercompiled in an abstract
environment.

Tying the knot.
Each time the focus changes to a residual, we add the focus to the store with
a name. If there's already a function there, we replace the residual with a
call to that function.

}



\section{A supercompiler for JavaScript}
\label{sec:js}

To demonstrate that our framework is practical and can be used for 
realistic languages, we implemented a supercompiler for JavaScript.

We started with an interpreter written as a CESK-like abstract machine.
To parse JavaScript we 
used the Nashorn JavaScript parser included in JDK 8.
The interpreter handles most of JavaScript.
Some of the more esoteric corners of the language are not implemented, but
they are nevertheless supported by the supercompiler.
The \c{eval} is supported by parsing its argument
using the Nashorn parser
then evaluating the resulting abstract syntax tree.

JavaScript is a notoriously difficult language to implement, with a complex
semantics for even the simplest operations~\cite{ecma,lambda-js}. However, one advantage of
implementing a partial evaluator versus a interpreter is that the difficult
cases can simply be residualized: that is, rather than implement the precise semantics
for some language features, we allow the evaluator to get stuck, forcing a split and later
residualization. In a similar vein, we do not implement many JavaScript primitives,
instead leaving calls to these primitives in the residual program.

Unfortunately, we did not have time to 
run performance measurements on the supercompiler output.
These will be reported in the final version of the paper.


\section{Related work}
\label{sec:related}

We are aware of one other toolkit for building supercompilers in Scala. MRSC~\cite{mrsc}
provides a framework based on a graph representation.
The supercompiler steps from one graph configuration to another.
The framework is more general than our approach, supporting different types
of supercompilation, in particular multi-result supercompilation, but
provides a lower-level interface than our framework.
In our framework, the user must only consider source language states,
and sequences of these states, and not a separate graph model.

The work most similar to ours is Bolingbroke
and Peyton Jones's work on building a practical supercompiler for
Haskell~\cite{supercompilation-by-eval}.
They describe building a supercompiler for
a call-by-need language (essentially core Haskell).
The supercompiler interleaves driving and splitting phases.
When driving cannot take a step, splitting is performed and driving is resumed if possible.
The evaluator is specified as a CEK-like abstract machine.
Our work differs in that it abstracts over the language being supercompiled.
It would be interesting to implement their abstract machine interpreter in our framework.




\section{Conclusions and future work}
\label{sec:conclusions}

Our goal with the framework is to make 
it possible to implement efficient supercompilers for arbitrary programming languages.
The framework takes care of most details of generalization and the whistle, leaving
the programmer free to focus on the programming language implementation.

We are working to better support residualization in the framework.
Beyond writing an interpreter for the subject language,
the user must
write code to split and later reassemble states.
It turns out that this code is often rather subtle.
We did not want to constrain the framework to use a particular representation
of terms in the subject language, however, use of a zipper representation
for terms
has been shown~\cite{taming-code-explosion} to be a good representation.
It may be possible to 
automatically construct splitting and reassembly code
using generic programming approaches, for instance
using lenses from the Shapeless library~\cite{shapeless}.

Beyond JavaScript, the framework is being used to construct supercompilers 
for other languages under development in our research group.
We are 
extending the framework to support other kinds of supercompilation,
including nondeterministic and multi-result supercompilation~\cite{mrsc,mrsc1}.
% Since our JavaScript interpreter is based on abstract machines, we are
% also looking at using the framework to implement static analyses.
Since our supercompiler framework is itself implemented as an abstract machine
interpreter, naturally, we should be supercompiling the supercompiler itself.

Source code for the framework is available at \url{http://github.com/nystrom/scsc}.




%% Acknowledgments
% \begin{acks}                            %% acks environment is optional
% \end{acks}


%% Bibliography
\bibliography{main}


%% Appendix
% \appendix
% \section{Appendix}

% Text of appendix \ldots

\end{document}
