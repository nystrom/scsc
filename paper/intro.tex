\newcommand\R[1]{\langle\!\langle#1\rangle\!\rangle}

\section{Introduction}
\label{sec:intro}

Partial evaluation is the well-known 
idea of specializing a program on some of its inputs~\cite{jones}.
Supercompilation is a partial evaluation technique that
attempts to evaluate programs as much as possible
at compile time~\cite{turchin}.
The name \emph{supercompilation} is a portmanteau of supervisor and compiler.
The supercompiler works by monitoring the execution of a (partial) evaluator.
When evaluation cannot proceed from a given program state, the
state is \emph{split} and evaluation is performed recursively. The
supercompiler conservatively identifies states from which continued evaluation
and splitting will not terminate and \emph{generalizes}
these states, ensuring the supercompiler terminates and 
producing a residual program
that (hopefully) implements the original program more efficiently.

Supercompilation has received considerable attention in the functional
programming 
literature because it subsumes common optimizations like deforestation~\cite{
  supercompilation-by-eval,supercompilation-blah}.
However, the design and implementation of a practical supercompiler is
elusive.
It is
not yet in mainstream use in any compiler because of the following
practical considerations: (1) supercompilation has a large compile time
compared to traditional optimization, and (2) supercompilation can generate
exponentially large code.
Attempts to address these issues have been made in
other work \cite{taming-code-explosion}, but not yet applied to 
mainstream languages such as Haskell or Scala.
To address these issues, more experiments need be done with different
supercompilation algorithms.

To further this research,
we describe a Scala framework for implementing 
partial evaluators and supercompilers.
To build a supercompiler for a given language, the programmer
need only provide the following:
\begin{enumerate}
\item Abstract syntax trees for the terms of the language.
\item An interpreter implemented as small-step rewrite rules
between equivalent program states. 
The interpreter can be implemented by term rewriting or as
an abstract machine~\cite{danvy,cesk}, allowing, for instance,
mutable variables or unstructured control-flow to be more easily supported.
\item Extensions of the interpreter to 
support \emph{residual states}. These are states that represent 
the residual program being constructed.
Stepping a residual state produces another residual state, and
ultimately results in the residual program.
\item A \emph{split} function to decompose a ``stuck'' program state into
one or more other states.
This function is used, for instance, when branching on an abstract value,
to return two states one which assumes the value was true, the other false.
After these two states are evaluated, the stuck state can be reassembled
into a rebuilt state from which more evaluation can be done.
\item A \emph{generalization} function 
that takes two states, producing a new state that generalizes the two.
\end{enumerate}

The framework provides the supercompilation algorithm and data structures
as well as code for detecting possible nontermination,
performing variable substitutions, and generalizations. Framework
code can be used as is, but supercompiler instances can override
the framework implementation for more precision. 

\paragraph*{Outline}
The rest of this paper is organized as follows.
Section~\ref{sec:nutshell} sketches how supercompilation works using
a small example.
In the Section~\ref{sec:framework}, we describe the interface
and implementation of our supercompiler framework.
Section~\ref{sec:cesk} describes the process of extending an abstract machine interpreter
to construct a supercompiler.
We use our framework to implement a supercompiler for JavaScript,
described in Section~\ref{sec:js}.
% Section~\ref{sec:experiments} describes some experiments.
Related work is discussed in Section~\ref{sec:related}.
Finally, we conclude in Section~\ref{sec:conclusions} with a discussion of
future work.

\section{Supercompilation in a nutshell}
\label{sec:nutshell}

To get an idea of what supercompilation does and how it works, consider the
\c{map} function on lists.\footnote{Examples are presented in Scala syntax.
The map example is due to \citeN{supercompilation-by-eval}.}
\begin{verbatim}
  def map[A,B](f: A => B, as: List[A]): List[A] =
    as match {
      case Nil => Nil
      case b::bs => f(b)::map(f, bs)
    }
\end{verbatim}

Consider the call \c{map(_ + 1, zs)}.
Supercompiling this call constructs a specialized version of \c{map}
that increments each element of the list by 1.

Evaluation is called \emph{driving} in the supercompiler literature.
We illustrate driving this expression by showing the state of the evaluator
as a pair $(e; k)$. The expression $e$ is the expression being
evaluated and $k$ is the continuation of the expression.
The initial state of the evaluation is the pair of the expression and the
empty continuation:
  \[
    \c{map(\_+1, zs)};~k = \bullet
  \]
Since the list argument is statically unknown, 
we pass an \emph{abstract value} into the 
\c{map} function.
To keep the presentation simple, we substitute the actual arguments for the
formal parameters.
  \[
    \begin{array}{l}
      \c{zs~match~\{} \\ 
    \c{~~~case~Nil \Rightarrow Nil}  \\
      \c{~~~case~b::bs \Rightarrow  (b+1)::map(\_+1, bs)~\}};~k = \bullet
  \end{array}
  \]
The body of \c{map}
performs a pattern match on the abstract value \c{zs}).
However, since \c{zs} is not statically known, the evaluator has reached an impasse.
Here the supercompiler steps in and 
\emph{splits} the evaluation state into two: one where \c{zs} is \c{Nil}, the
other where it is not.

Assuming \c{zs} is \c{Nil}, we arrive immediately in this state:
    \[
      \c{Nil};~k = \bullet
    \]
Here driving stops.
The supercompiler produces a residual expression $\R{\c{Nil}}$.
we write
$\R{e}$ for a residual expression $e$ to be output 
by the supercompiler. For purposes of
this example, residual expressions may be considered values.
Applying a continuation to a residual expression produces a larger residual
expression.\footnote{In
practice, residual expressions as values
must be treated a \emph{linear} values to avoid work duplication.
For instance, we cannot simply substitute a residual expression
for a function parameter since that would duplicate the expression
for each occurrence of the parameter in the function body.}

On the other hand, if \c{zs} is not \c{Nil}, we drive as follows:
    \begin{align*}
       \c{(b+1)::map(\_+1, bs)};~& k = \bullet \\
       \downarrow&\\
       \c{(b+1)};~& k = \c{\bullet::map(\_+1, bs)} \\
       \downarrow&\\
       \c{b};~& k = \c{(\bullet+1)::map(\_+1, bs)} 
    \end{align*}
Here again driving is stuck because \c{b} is unknown.
Unlike before, we cannot split the evaluator state, instead we \emph{rebuild},
generating a residual
expression $\R{\c{b}}$:
       \[
       \R{\c{b}};~ k = \c{(\bullet+1)::map(\_+1, bs)}
     \]
After continued driving:
    \begin{align*}
       \R{\c{b+1}};~& k = \c{\bullet::map(\_+1, bs)} \\
       \downarrow&\\
       \c{map(\_+1, bs)};~& k = \c{\R{b+1}::\bullet} 
    \end{align*}
the supercompiler arrives in a state where
the focus expression is a call to \c{map}.
Indeed, the arguments are a renaming of the original function arguments.
At this point, the supercompiler \emph{folds}.
It generates a function stub that generalizes the two calls:
\begin{verbatim}
  def map1(as: List[A]): List[A] = ...
\end{verbatim}
and replaces the focus expression with a residual call to that function.
To construct the body of the new function, it continues driving the residual
expression.
    \begin{align*}
      \c{\R{map1(bs)}};~& k = \c{\R{b+1}::\bullet} \\
       \downarrow&\\
      \c{\R{(b+1)::map1(bs)}};~& k = \bullet 
    \end{align*}
Here again, driving has stopped.
After splitting on \c{zs}, the evaluator has produced two residual expressions
\c{Nil} and $\c{\R{(b+1)::map1(bs)}}$.
The supercompiler reassembles the pattern match from the split states and produces 
the final residual program:
\begin{verbatim}
  def map1(as: List[A]): List[A] =
    as match {
      case Nil => Nil
      case b::bs => (b+1)::map1(bs)
    }

  map1(zs)
\end{verbatim}

\section{Our framework}
\label{sec:framework}

Our focus is on building supercompilers for program specialization and
optimization. Supercompilation has been used in other applications, in
particular for theorem proving and function inversion~\cite{mrsc}.
These applications have different tradeoffs. For instance, in theorem proving
applications, the program being supercompiled is 
typically a pure functional program with call-by-name semantics,
and efficiency and size of the generated code is often not a big concern.


\subsection{The interpreter}

To implement a supercompiler for a given language, the user must provide an
interpreter for the language. The framework requires the interpreter be
implemented by stepping from one program state to another. This implementation
supports different modes of interpretation, in particular term rewriting
and abstract machines.

An interpreter implements the \c{Interpreter} trait shown in Figure~\ref{fig:interpreter}.
The interface is functional: the state of the interpreter should 
be maintained only in a \c{State} value, on which the trait is parameterized.
The trait requires a \c{step} method
that advances the interpreter to the next state.
Importantly, 
this method need not be total:
the interpreter can get ``stuck'' in a state because of missing
information. If the interpreter becomes stuck, the supercompiler can
still make further progress, as outlined below.
Indeed, we take advantage of this property to implement
a supercompiler for JavaScript. Indeed, our JavaScript interpreter does not
support all features of JavaScript, but using our framework,
any JavaScript program can be supercompiled.
A more complete interpreter implementation leads to
a more effective supercompiler, however, completeness is not required.

When the \c{step} method returns \c{None}, the supercompiler calls \c{split}
on the current interpreter state.
If successful, \c{split} returns a list of states the supercompiler should 
attempt to drive. It also returns an \c{Unsplitter}, which is a function
that attempts to recombine the results of the split states.

If both \c{step} and \c{split} fail, the supercompiler calls \c{rollback}.
This takes the history of states the supercompiler executed (up to the
previous split), and returns a new history, which replaces the old.
This Orwellian rewriting of history is needed by some interpreter
implementations to help construct residual programs in later states.

The \c{fold} method attempts to fold the current state (the last
state in the history) with a previous state. A new generalized state
is returned. Fold is called before each attempt to \c{step}.

Finally the \c{whistle} function checks for possible nontermination
of a history.
If it returns \c{true} (``the whistle blows''), the supercompiler stops driving as if \c{step}
returned \c{None}. Our framework provides several whistle implementations
that can be used by the interpreter. Different whistle implementations are
a topic of ongoing research~\cite{whistles64}.

\begin{figure*}
  \begin{verbatim}
trait Interpreter[State] {
  // Step to the next state, if possible.
  def step(s: State): Option[State]

  // Split the current state, if possible.
  def split(s: State): Option[(List[State], Unsplitter[State])]

  type History = List[State]

  // Replace the history with a new history. Returns Nil on failure.
  def rollback(h: History): History

  // Fold the current state (the last state in the history)
  // with a state in the history. If successful, returns a
  // new state that generalizes the two folded states.
  def fold(h: History): Option[State]

  // Check if we should stop driving.
  def whistle(h: History): Boolean
}
  \end{verbatim}
  \caption{Interpreter trait}
  \label{fig:interpreter}
\end{figure*}

\subsection{The supercompilation algorithm}

We present supercompilation itself as evaluation on an abstract machine.
The states of the machine are shown in Figure~\ref{fig:sc-data}.
To distinguish between supercompiler states and interpreter state we refer
to the latter as metastates.

Each metastate contains the current interpreter state, a (possibly truncated or
revised) history of previous states and a continuation. 
There are four kinds of metastate: \c{Drive}, \c{Split}, \c{Rebuild},
and \c{Done}.
Continuations are stacks of \c{Meta}\-\c{Frame}. The continuation manages the
splitter of the supercompiler.
The \c{RunSplit} frame
contains the list of unprocessed split states, completed split histories,
as well as the root state that caused the split, its history, and an \c{Unsplitter}
function that takes the completed split histories and attempts to rebuild
the root state.

\begin{figure*}
  \begin{verbatim}
type History = List[State]
type MetaCont = List[MetaFrame]

sealed trait MetaState
case class Drive(state: State, history: History, k: MetaCont) extends MetaState
case class Split(state: State, history: History, k: MetaCont) extends MetaState
case class Rebuild(state: State, history: History, k: MetaCont) extends MetaState
case class Done(state: State) extends MetaState

sealed trait MetaFrame
case class RunSplit(pending: List[State], complete: List[History],
                    root: State, rootHistory: History,
                    unsplit: Unsplitter[State]) extends MetaFrame
  \end{verbatim}
  \caption{Supercompiler data types}
  \label{fig:sc-data}
\end{figure*}

The supercompiler implementation is shown in Figure~\ref{fig:sc}.
The \c{sc} function takes the current metastate and a history of
metastates, computes the next metastate and then calls itself recursively
if the metastate is not \c{Done}.

If the metastate is \c{Drive}, the supercompiler tries to fold
the current history. If successful it enters the \c{Rebuild}
metastate with the new folded history.
Otherwise, if the whistle blows, indicating 
the current interpreter state might lead to nontermination,
the supercompiler enters the \c{Split} metastate.
Otherwise, the supercompiler tries to take a step, splitting on failure.

In the \c{Split} metastate, the supercompiler attempts to split
the current interpreter state.
If successful, it \c{Drives} the first split state, pushing a continuation to
evaluate the other split states.
On failure, the supercompiler \c{Rebuilds}.

In the \c{Rebuild} metastate, the supercompiler
either starts driving the next pending split state 
or, if the split is complete, attempts to reassemble the root
state of the split. Three outcomes are possible: either a
new root state is produced, or the unsplit function forces
another split, or the unsplit fails.
In the last case, the supercompiler attempts to rollback
the current history. If the rollback succeeds, driving continues,
otherwise the split (meta)frame is popped and 
the root of the split is rebuilt.

If in the \c{Rebuild} state and the continuation is empty,
a rollback is attempted. If the rollback fails, the supercompiler
terminates.

\begin{figure*}
  \begin{verbatim}
def sc(ms: MetaState, mhistory: List[MetaState]): MetaState = {
  val next = ms match {
    // Try to drive
    case Drive(s, h, k) =>
      interp.fold(s::h) match {
        case Some(s1::h1) =>
          Rebuild(s1, h1, k)   // fold successful
        case _ if (interp.whistle(s::h) || metaWhistle(ms::mhistory)) =>
          Split(s, h, k)
        case _ =>
          interp.step(s) match {
            case Some(s1) => Drive(s1, s::h, k)
            case None     => Split(s, h, k)     } }

    // Try to split
    case Split(s, h, k) =>
      interp.split(s) match {
        case Some((first::rest, unsplit)) if ! metaWhistle(ms::mhistory) =>
          Drive(first, Nil, RunSplit(rest, Nil, s, h, unsplit)::k)
        case _ => Rebuild(s, h, k) }

    // Run the next split state
    case Rebuild(s, h, RunSplit(next::pending, complete, s0, h0, unsplit)::k) =>
      Drive(next, Nil, RunSplit(pending, complete :+ (s::h), s0, h0, unsplit)::k)

    // All the split states are done. Try to reassemble
    case Rebuild(s, h, RunSplit(Nil, complete, s0, h0, unsplit)::k) =>
      unsplit(complete :+ (s::h)) match {
        case UnsplitOk(s1) =>
          Drive(s1, h0, k)
        case Resplit(states, unsplit) if metaWhistle(ms::mhistory) =>
          Rebuild(s0, h0, k)
        case Resplit(first::rest, unsplit) =>
          Drive(first, Nil, RunSplit(rest, Nil, s0, h0, unsplit)::k)
        case _ =>
          interp.rollback(s::h) match {
            case s1::h1 => Drive(s1, h1, k)
            case Nil    => Rebuild(s0, h0, k)   }

    case Rebuild(s, h, Nil) =>
      interp.rollback(s::h) match {
        case s1::h1 => Drive(s1, h1, Nil)
        case Nil    => Done(s) }

    case Done(s) =>
      // Finished
      return Done(s)
  }

  sc(next, ms::mhistory) // Keep going
}
  \end{verbatim}
  \caption{Supercompiler algorithm}
  \label{fig:sc}
\end{figure*}


\subsection{The whistle}

In the supercompilation literature, the \emph{whistle}
is a subroutine that detects if the supercompiler
might not terminate.
This can happen, of course, when evaluating an actual infinite loop, but also 
when evaluating loops where the loop condition is statically unknown, or
when evaluating recursive calls. Indeed nontermination occurs more often
during partial evaluation than normal evaluation because of missing
information.

If the whistle blows, the supercompiler
stops driving and splits the current state.
Our supercompiler uses two whistles: the first operates
on interpreter states (implemented by \c{Interpreter.whistle})
and ensures termination of driving. The second operates on states of the supercompiler
itself and ensures termination of splitting.

Whistles are an ongoing topic of research~\cite{whistles64}.
Typically, the whistle is based on an \emph{well-quasi-ordering} (WQO) of
interpreter states.
Indeed, any WQO on states can be used as a whistle.
States of the supercompiler are stored in a history.
When a new state is entered, it is compared to previous states
in the history. If the new state is greater than or equal a previous state,
according to the WQO, then
there is a potential infinite sequence of states
the supercompiler will pass through. The supercompiler halts driving or
splitting to avoid running forever.

Various WQOs have been used in the supercompiler literature.
Typically, they are based on a homeomorphic embedding of
states~\cite{whistles64}.
Our framework provides several whistles: a whistle that
never blows (thus allowing nontermination), a whistle that 
blows when the state size becomes too large or when the history of states becomes too
long.
The default whistle provided by the framework uses a \emph{tag-bag}.
In our implementation, 
the tag-bag of a state is a bag (multiset) of 
Scala class names in the state.
One state is less than another if it has the same set of tags
in the bag, but the bag is smaller.
\begin{align*}
  s \preceq s'
   \iff&
  \mathit{set}(\mathit{tagbag}(s))
  =
  \mathit{set}(\mathit{tagbag}(s')) \\
  &
  \wedge
  |\mathit{tagbag}(s)|
  \le
  |\mathit{tagbag}(s')|
\end{align*}

The implementation works for arbitrary state implementations, using the Kiama
library~\cite{kiama} to traverse data structures, collecting class names.
Users of the framework can override the tagging operation for more precision,
for instance by including constant values occurring in the state as additional tags.

\subsection{Folding and generalization}

Another component of the supercompilation is the \c{fold} function.
Like the whistle,
folding compares the current state with previous states in the history.
If two states are identical up to renaming, the states can be \emph{generalized}.
For instance, in the example from Section~\ref{sec:nutshell},
the \c{map} function is called initally as
\c{map(\_ + 1, zs)}
and then later
\c{map(\_ + 1, bs)}.
When supercompilation is complete the residual of the first call (with \c{zs})
will be the body of a function that maps \c{\_ + 1} over the free
variable \c{zs}.
We can generalize these two calls by introducing a new function
that takes \c{zs} as a parameter. 
Folding replaces the call in the current state
with a call to the new function, and then builds the residual function body.
The history is rolled back to the previous state, with the original call
replaced with a call to the new function.

The framework provides methods for generalizing terms.
A generalization of two terms
replaces differing subterms of the two terms with a variable.
The \emph{most-specific generalization}
(MSG) 
performs the fewest substitutions possible~\cite{supercompilation-blah}.

\begin{verbatim}
trait MSG[Term, Var <: Term] {
  type Subst = Map[Var, Term]
  def freshVar: Var
  def msg(t1: Term, t2: Term):
    Option[(Term, Subst, Subst)] = ...
}
\end{verbatim}

To use this functionality, the user implements the \c{MSG} trait
providing a method to generate fresh variable terms.
The \c{msg} method returns not only the MSG term, but also two substitutions, one 
that when applied to the MSG returns the first term \c{t1}
and the other that when applied to the MSG returns \c{t2}.
The framework provides methods for building and applying substitutions.
The MSG \emph{always} exists, but the \c{msg} method fails if the MSG
is just a variable, meaning the head of the two terms
is of a different shape.
If the two terms are identical upto renaming, they are
required to be folded. This is one of the key distinctions between supercompilation and simple
partial evaluation.

